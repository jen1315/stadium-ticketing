\documentclass{article}

% Language setting
% Replace `english' with e.g. `spanish' to change the document language
\usepackage[italian]{babel}

% Set page size and margins
% Replace `letterpaper' with `a4paper' for UK/EU standard size
\usepackage[a4paper,top=2cm,bottom=2cm,left=3cm,right=3cm,marginparwidth=1.75cm]{geometry}

% Useful packages
\usepackage{multirow}
\usepackage{parskip}
\usepackage{setspace}
\usepackage{graphicx}
\usepackage[colorlinks=true, allcolors=blue]{hyperref}

\begin{document}

\begin{titlepage} 
\begin{center}
    \thispagestyle{empty}
    \begin{spacing}{2.0}
        \Large{\textbf{UNIVERSITÀ DEGLI STUDI DI PADOVA}\\DIPARTIMENTO DI MATEMATICA\\Informatica\\}
    \end{spacing}
    \includegraphics[width=5.5cm]{img/UNIPD_Logo.png}
    \begin{spacing}{2.0}
        \large{\textbf{PROGETTO DI TECNOLOGIE WEB}}
    \end{spacing}
    \begin{spacing}{1.0}
        \Large{{Stadium ticketing}}
    \end{spacing}
    \begin{abstract}
        La relazione delle specifiche del progetto di Tecnologie Web: sito di uno stadio che ospita numerosi eventi di cui si possono acquistare i biglietti selezionando il posto.
    \end{abstract}
    \begin{table}[h!]
        \centering
        \begin{tabular}{|c|cc|}
            \hline
            Ruolo & Username & Password \\ \hline\hline
            gestore & admin & admin \\ \hline
            \multirow{2}{3em}{utente} & user & user \\
            & user1 & user1 \\ \hline
        \end{tabular}
        \caption{Tabella contenente le informazioni di login già esistenti per test.}
        \label{table:1}
    \end{table}

    \vfill
    \begin{spacing}{1.0} 
        \raggedright{\large{Qiao Qiao Cai (2111010)\\}}
        \normalsize{\large{Samuele Vendramin (2111934)\\}}
        \normalsize{\large{Stefano Longhena (2111936)\\}}
        \normalsize{\large{Mattia Nicastro (2111024)\\}}
    \end{spacing}
    \vspace{2.0cm}
    \large{ANNO ACCADEMICO 2025/2026}   
\end{center}
\end{titlepage}
\newpage
\tableofcontents
\newpage

\section{Introduzione}
Il nostro gruppo ha deciso di progettare un sito di uno stadio che offre l'opportunità di acquistare biglietti agli eventi che ospitano. Per ogni biglietto è possibile selezionare un posto, il quale cambia di costo a seconda della sezione in cui si trova.
Ogni evento ha una pagina che la descrive
\subsection{Obiettivo}


\section{Utenza}
Sono descritti in questa sezione i casi d'uso dei diversi utenti che visitano il sito. Saranno specificati le funzionalità ai quali hanno accesso e le informazioni che possono visualizzare.
\par
Sono presenti nella tabella \ref{table:1} le informazioni di login già esistenti nel database.

\subsection{Visitatori}
Tutti coloro che visitano la pagina hanno accesso alla home, la pagina degli eventi, la pagina di login e quella di registrazione. 

\subsection{Amministratore}
Utente con accesso a pagine speciali di gestione del sito.

\subsection{Utenti registrati}
Gli utenti registrati nel sito devono eseguire l'accesso alla piattaforma tramite la pagina di login per visualizzare la loro zona personale. Questo include la pagine degli biglietti acquistati e il profilo.

\section{Design}
\subsection{Colori}
\subsection{Font}
\section{Struttura}
\subsection{HTML}
\subsection{CSS}
Definito nel file \verb|style.css|. E' il foglio di stile
\subsection{PHP}
\subsection{Database}
\section{Funzionalità}
\subsection{Navigazione a tab}
\end{document}
